\documentclass[10pt]{beamer}
\usepackage[utf8]{inputenc}
\usepackage{graphicx}
\usepackage {mathtools}
\usepackage{utopia}
\usetheme{CambridgeUS}
\usecolortheme{dolphin}

\definecolor{myNewColorA}{RGB}{126,12,110}
\definecolor{myNewColorB}{RGB}{165,85,154}
\definecolor{myNewColorC}{RGB}{203,158,197}
\setbeamercolor*{palette primary}{bg=myNewColorC}
\setbeamercolor*{palette secondary}{bg=myNewColorB, fg = white}
\setbeamercolor*{palette tertiary}{bg=myNewColorA, fg = white}
\setbeamercolor*{titlelike}{fg=myNewColorA}
\setbeamercolor*{title}{bg=myNewColorA, fg = white}
\setbeamercolor*{item}{fg=myNewColorA}
\setbeamercolor*{caption name}{fg=myNewColorA}
\usefonttheme{professionalfonts}
\usepackage{natbib}
\usepackage{hyperref}

\titlegraphic{\includegraphics[height=2.3cm]{téléchargement.png}}

\setbeamerfont{title}{size=\large}
\setbeamerfont{subtitle}{size=\small}
\setbeamerfont{author}{size=\small}
\setbeamerfont{date}{size=\small}
\setbeamerfont{institute}{size=\small}
\title{Utilisation des Technologies de
l’Information \\ et de la
Communication pour la
modélisation \\et la diffusion des
résultats d’une équation\\ aux
dérivées partielles}

\subtitle{Projet TIC}

\author{ZIOUCHE Tiziri,GUERNANE Imene}
\institute{ Usthb}
\date{Décembre 2025}


\AtBeginSection[]{
  \begin{frame}
  \vfill
  \centering
  \begin{beamercolorbox}[sep=8pt,center,shadow=true,rounded=true]{title}
    \usebeamerfont{title}\insertsectionhead\par
  \end{beamercolorbox}
  \vfill
  \end{frame}
}
\begin{document}
\frame{\titlepage}
\begin{frame}

\frametitle{Table des matières}
\label{sec:TOC}
\tableofcontents
\end{frame}


\begin{frame}{Introduction}
    \begin{itemize}
        \item Les Technologies de l'Information  et de la Communication (TIC) sont essentielles au développement scientifique et technologique.\pause
        \item Leur évolution rapide a transformé l’analyse, la modélisation et la diffusion des connaissances dans divers domaines, notamment en mathématiques appliquées et en physique.\pause
        \item Les outils numériques permettent désormais de simuler, visualiser et interpréter des phénomènes avec précision, remplaçant ainsi des calculs analytiques complexes ou inaccessibles.\pause
        \item Un domaine clé de l’utilisation numérique est l’étude des équations aux dérivées partielles (EDP), qui modélisent des phénomènes naturels comme la propagation de la chaleur ou la diffusion d’ondes.\pause
        \item La résolution numérique de ces équations est souvent nécessaire, car la solution analytique est rarement disponible.\pause
        \item Ce projet démontre l’application des TIC pour modéliser, simuler, analyser et diffuser des résultats scientifiques d’un phénomène physique.
    \end{itemize}
\end{frame}

\begin{frame}{Introduction}
    \begin{itemize}
        \item LaTeX a été utilisé pour rédiger ce rapport de manière professionnelle.\pause
        \item GitHub a servi à organiser et stocker le code source, garantissant transparence et reproductibilité.\pause
        \item La collaboration scientifique a eu lieu via la plateforme Overleaf.\pause
        \item Le projet se concentre sur la résolution numérique de l’équation de la chaleur à l’aide de la méthode des différences finies.\pause
        \item MATLAB a été choisi pour la simulation en raison de ses performances de calcul et de visualisation graphique.\pause
        \item Ce travail combine théorie mathématique et application numérique, reliant les bases des EDP à leur mise en œuvre informatique.
    \end{itemize}
\end{frame}

\section{Technologies de l'Information et de la Communication (TIC)}

\begin{frame}{Introduction aux TIC}
    \begin{itemize}
        \item Les TIC regroupent l’ensemble des outils, logiciels, plateformes et méthodes permettant de produire, traiter, analyser, stocker et partager l’information. \pause
        \item Elles jouent un rôle fondamental dans l’enseignement supérieur, la recherche scientifique et l’ingénierie. \pause
        \item Elles facilitent l’accès à l’information, améliorent le travail collaboratif et permettent de réaliser des simulations complexes.\pause
        \item Les TIC peuvent être classés en plusieurs catégories:
    \end{itemize}
    \cite{TIC_LocalGlobal}
\end{frame}

\begin{frame}{Outils de résolution numérique}
    \begin{itemize}
        \item Indispensables pour traiter les équations aux dérivées partielles (EDP), effectuer des simulations et visualiser des phénomènes physiques. \pause
        \item Les outils les plus utilisés incluent : MATLAB, Python, GeoGebra et Jupyter Notebook.
    \end{itemize}
\end{frame}

\begin{frame}{MATLAB}
    \begin{itemize}
        \item Environnement de calcul numérique largement utilisé en ingénierie et sciences appliquées. \pause
        \item Permet d’implémenter facilement des schémas numériques (différences finies, éléments finis, etc.). \pause
        \item Produit des graphiques clairs et précis pour l’analyse des EDP.
    \end{itemize}
    \cite{MathWorksSolutions}
\end{frame}

\begin{frame}{Python}
    \begin{itemize}
        \item Langage open-source flexible avec des bibliothèques scientifiques puissantes (NumPy, SciPy, Matplotlib). \pause
        \item Permet de coder rapidement des méthodes numériques pour les EDP. \pause
        \item Alternative gratuite et performante à MATLAB, avec solutions évolutives et personnalisables.\cite{PythonAbout}
    \end{itemize}
\end{frame}

\begin{frame}{GeoGebra}
    \begin{itemize}
        \item Outil interactif pour la visualisation géométrique et mathématique. \pause
        \item Moins adapté aux simulations avancées, mais utile pour illustrer des concepts liés aux EDP. \pause
        \item Permet de visualiser conditions limites et propagation de solutions.\cite{GeoGebraHome}
    \end{itemize}
\end{frame}

\begin{frame}{Jupyter Notebook}
    \begin{itemize}
        \item Environnement interactif pour exécuter du code Python. \pause
        \item Idéal pour structurer des calculs, documenter la résolution et partager les résultats. \pause
        \item Combine texte, équations et graphiques, parfait pour l’enseignement et les démonstrations.\cite{JupyterAbout}
    \end{itemize}
\end{frame}

\subsection{Outils de collaboration scientifique}

\begin{frame}{Outils de collaboration scientifique}
    \begin{itemize}
        \item Ces outils facilitent le travail à distance, le partage de documents scientifiques et la gestion des références bibliographiques. \pause
        \item Ils sont essentiels pour la rédaction collaborative et la diffusion des résultats scientifiques.
    \end{itemize}
\end{frame}

\begin{frame}{Overleaf}
    \begin{itemize}
        \item Plateforme en ligne dédiée à l’édition de documents LaTeX. \pause
        \item Permet un travail collaboratif en temps réel. \pause
        \item Idéal pour rédiger rapports, articles et mémoires scientifiques.\cite{OverleafFeatures}
    \end{itemize}
\end{frame}

\begin{frame}{Google Colab}
    \begin{itemize}
        \item Environnement cloud pour exécuter du code Python sans installation locale. \pause
        \item Utilisé pour les calculs, simulations et partage de notebooks. \pause
        \item Facilite la collaboration entre étudiants et chercheurs.\cite{GoogleColab}
    \end{itemize}
\end{frame}

\begin{frame}{Zotero}
    \begin{itemize}
        \item Logiciel de gestion bibliographique. \pause
        \item Permet de collecter, organiser et citer automatiquement les références. \pause
        \item Compatible avec LaTeX, Word et d’autres formats.\cite{ZoteroHome}
    \end{itemize}
\end{frame}




\subsection{Espaces numériques de travail}

\begin{frame}{GitHub}
    \begin{itemize}
        \item Plateforme de gestion de versions pour stocker, partager et collaborer sur des projets informatiques. \pause
        \item Dans le cadre des EDP et des simulations numériques : sauvegarde du code MATLAB/Python, suivi des modifications et travail en groupe efficace.\cite{WhyGitHub}
    \end{itemize}
\end{frame}

\subsection{Outils de présentation}

\begin{frame}{PDF}
    \begin{itemize}
        \item Format standard de diffusion scientifique. \pause
        \item Garantit une mise en page fixe et une compatibilité universelle.
    \end{itemize}
\end{frame}

\begin{frame}{LaTeX interactif}
    \begin{itemize}
        \item Permet d’intégrer animations, graphiques dynamiques et résultats numériques interactifs dans des documents PDF. \pause
        \item Idéal pour des documents avancés et interactifs.
    \end{itemize}
\end{frame}

\begin{frame}{Beamer}
    \begin{itemize}
        \item Classe LaTeX pour créer des présentations professionnelles. \pause
        \item Appréciée dans les conférences scientifiques pour intégrer équations et figures complexes.
    \end{itemize}
\end{frame}

\begin{frame}{PowerPoint}
    \begin{itemize}
        \item Outil populaire pour les présentations classiques. \pause
        \item Simple à utiliser et permet de créer des supports visuels rapides et efficaces.
    \end{itemize}
\end{frame}

\begin{frame}{Canva}
    \begin{itemize}
        \item Plateforme de design en ligne pour créer des présentations modernes et esthétiques. \pause
        \item Utile pour concevoir affiches scientifiques, posters et diapositives visuellement attractives.\cite{CanvaFeatures}
    \end{itemize}
\end{frame}

\subsection{Intérêt des TIC dans la recherche scientifique}

\begin{frame}{Rôle central des TIC}
    \begin{itemize}
        \item Les TIC occupent une place centrale dans la recherche moderne. \pause
        \item Elles offrent un accès rapide à l’information scientifique : bases de données, articles, revues, résultats de recherche. \pause
        \item Elles permettent l’automatisation des calculs complexes. \pause
        \item Elles facilitent la collaboration entre chercheurs via des plateformes en ligne. \pause
        \item Elles améliorent la qualité de présentation des rapports, articles et communications scientifiques. \pause
        \item Elles assurent la reproductibilité des résultats grâce au partage du code et des données.
    \end{itemize}
\end{frame}

\begin{frame}{Importance des TIC pour les EDP}
    \begin{itemize}
        \item Les EDP nécessitent souvent une résolution numérique, car les solutions analytiques sont limitées. \pause
        \item Les TIC interviennent à plusieurs niveaux dans l’étude et la résolution des EDP : 
    \end{itemize}
\end{frame}

\begin{frame}{Apports des TIC dans la résolution des EDP}
    \begin{itemize}
        \item Implémentation des méthodes numériques : différences finies, éléments finis, volumes finis. \pause
        \item Visualisation graphique des solutions numériques.\pause
        \item Gestion et stockage des projets de simulation via plateformes collaboratives comme GitHub. \pause
        \item Documentation structurée des modèles et résultats via LaTeX, Jupyter Notebook ou Overleaf. \pause
    \end{itemize}
Ces apports font des TIC un élément incontournable pour les études modernes sur les EDP.    
\end{frame}


\section{Étude de l'équation de la chaleur: cadre théorique}

\begin{frame}{Introduction}
    \begin{itemize}
        \item L’équation de la chaleur a été choisie pour les raisons suivantes: 
    \end{itemize}
\end{frame}

\begin{frame}{Intérêt de l'équation de la chaleur}
    \begin{itemize}
        \item Modèle physique simple mais riche mathématiquement, illustrant les notions fondamentales des EDP. \pause
        \item Facile à simuler numériquement sous MATLAB. \pause
        \item Structure parabolique adaptée aux méthodes numériques comme les différences finies. \pause
        \item Présente dans de nombreuses applications industrielles : refroidissement de matériaux, conduction thermique, électronique, thermique biomédicale, etc.
    \end{itemize}
\end{frame}

\subsection{Position du problème}

\begin{frame}{Formulation du problème}
    \begin{itemize}
        \item On considère l'équation de la chaleur sur l'intervalle $[0,L]$, avec un coefficient de diffusivité thermique $\alpha>0$. \pause
        \item On cherche une fonction $u(x,t)$ telle que :
    \end{itemize}
    \[
    \begin{cases}
    u_t(x,t)=\alpha\,u_{xx}(x,t),    x\in[0,L],\; t>0, \\[2mm]
    u(0,t)=0,\quad u(L,t)=0,         t\ge 0, \\[2mm]
    u(x,0)=f(x), \& x\in[0,L].
    \end{cases}   
    \] \pause
    \begin{itemize}
        \item Cette équation décrit la diffusion progressive de la chaleur dans une barre de longueur $L$ avec extrémités à température constante nulle et condition initiale $f(x)$. \cite{MemoireChaleur}
    \end{itemize}
\end{frame}

\subsection{Solution formelle}

\begin{frame}{Solution analytique classique}
    \begin{itemize}
        \item La solution par séparation des variables s’écrit sous forme de série de Fourier :
    \end{itemize}
    \[
    u(x,t) = \sum_{n=1}^{\infty} B_n \sin\left(\frac{n\pi x}{L}\right)
    \exp\left(-\alpha \left(\frac{n\pi}{L}\right)^2 t\right)
    \] \pause
    \begin{itemize}
        \item Les coefficients $B_n$ sont calculés à partir de la condition initiale $f(x)$ :
    \end{itemize}
    \[
    B_n = \frac{2}{L} \int_0^L f(x)\sin\left(\frac{n\pi x}{L}\right)\,dx
    \]
    \cite{MemoireChaleur}
\end{frame}


\section{Étude de l'équation de la chaleur: approche numérique}
\begin{frame}{Approche numérique de l'équation de la chaleur}
\begin{itemize}
    \item<+-> La résolution analytique des EDP n’est pas toujours possible.
    \item<+-> Une approche numérique est nécessaire pour obtenir des solutions exploitables.
    \item<+-> La méthode des différences finies est simple, efficace et très utilisée.
\end{itemize}
\end{frame}



\subsection{Présentation de la méthode numérique: méthode des différences finies}
\begin{frame}{Méthode des différences finies}
\begin{itemize}
    \item<+-> La méthode repose sur l'approximation des dérivées par des expressions discrètes.
    \item<+-> On démare de la discrétisation du domaine : $x_i = i\Delta x$, $t_n = n\Delta t$.
    \item<+-> Puis on utilise l'approximation temporelle : $\displaystyle u_t \approx \frac{u_i^{n+1}-u_i^n}{\Delta t}$.
    \item<+-> et l'approximation spatiale : $\displaystyle u_{xx} \approx \frac{u_{i+1}^n - 2u_i^n + u_{i-1}^n}{(\Delta x)^2}$.
    \item<+-> La substitution de ces approximations donne → schéma explicite FTCS.
\end{itemize}
\end{frame}



\begin{frame}{Schéma explicite FTCS}
\begin{itemize}
    \item<+-> Après approximation, on obtient :
\end{itemize}

\[
\frac{u_i^{n+1}-u_i^n}{\Delta t}
=
\alpha \frac{u_{i+1}^n - 2u_i^n + u_{i-1}^n}{(\Delta x)^2}
\]

\begin{itemize}
    \item<+-> Condition initiale : $u_i^0 = f(x_i)$.
    \item<+-> Conditions limites : $u_0^n = 0$, $u_N^n = 0$.
\end{itemize}
\end{frame}





\begin{frame}{Condition de stabilité}
\begin{itemize}
    \item<+-> Le schéma explicite nécessite une condition de stabilité.
    \item<+-> Coefficient de stabilité :
\[
r = \frac{\alpha\,\Delta t}{(\Delta x)^2}
\]
    \item<+-> Condition : $r < 0.5$.
    \item<+-> Cette contrainte garantit une simulation stable.\cite{FDM_Chaleur}
\end{itemize}
\end{frame}



\subsection{Simulation sous MATLAB}
\begin{frame}{Simulation sous MATLAB}
Le code pour obtenir les graphes exacte et approché est le suivant:
\begin{figure}[h]
    \centering
    \includegraphics[width=0.43\textwidth]{3} 
    \caption{Plot Exact vs Approché }
\end{figure}

\end{frame}


\begin{frame}{Visualisation}
L'exécution du code précédent en MATLAB donne les graphes superposés suivants:
    \begin{figure}[h]
    \centering
    \includegraphics[width=0.43\textwidth]{comparaison.png} 
    \caption{Graphe de comparaison}
\end{figure}
\end{frame}



\subsection{Interprétation des résultats}
\begin{frame}{Comparaison visuelle}
\label{sec:VDS}
\begin{itemize}
    \item<+-> Solution exacte : décroissance exponentielle.
    \item<+-> Solution approchée : profil très similaire.
    \item<+-> Superposition : courbes presque confondues.
    \item<+-> Le schéma numérique reproduit bien le comportement physique.
\end{itemize}
\end{frame}

\begin{frame}{Estimation de l'erreur numérique}
Le code utilisé pour obtenir l'estimation de l'erreur est le suivant:
\begin{figure}[h]
    \centering
    \includegraphics[width=0.7\textwidth]{4} 
    \caption{Estimation de l’erreur numérique }
\end{figure}

\end{frame}

\begin{frame}{Estimation de l’erreur numérique}
\begin{itemize}
    \item<+-> La comparaison entre la solution exacte et approchée se fait par le calcul suivants:
    \item<+-> Erreur ponctuelle : $e = u - u_{exact}$.
    \item<+-> Norme $L_2$ calculée : $6.03 \times 10^{-4}$.
    \item<+-> Norme $L_\infty$ calculée : $3.02 \times 10^{-3}$.
    \item<+-> L’erreur est faible → bonne précision du schéma.
\end{itemize}
\end{frame}


\section{Stockage des codes LaTeX \& MATLAB sur GitHub}

\begin{frame}{Gestion du code avec GitHub}
\label{sec:GITHUB}
    \begin{itemize}
        \item Un \href{https://github.com/Thiz-Thiz/Projet-Tic..git}{\textcolor{blue}{dépôt GitHub}} a été créé pour centraliser tous les fichiers du projet. \pause
        \item Permet la traçabilité des modifications et le partage avec les collaborateurs. \pause
        \item Structure du dépôt :
    \end{itemize}
    
  \begin{figure}[h]
    \centering
    \includegraphics[scale=0.7]{5.png} 
\end{figure}
\end{frame}

\begin{frame}{Avantages de GitHub}
    \begin{itemize}
        \item Collaboration entre plusieurs étudiants ou chercheurs. \pause
        \item Suivi des modifications. \pause
        \item Partage des codes et résultats avec transparence. \pause
        \item Intégration avec des plateformes CI/CD pour automatiser certaines simulations ou analyses.
    \end{itemize}
    \begin{figure}[h]
        \centering
        \includegraphics[width=0.8\textwidth]{Capture d’écran 2025-12-13 183507.png} 
        \caption{Interface GitHub du projet}
    \end{figure}
\end{frame}
\section{Interactions PDF \& inclusion d'une présentation}

\begin{frame}{Interactions dans le PDF}
  Cette section concerne le PDF dont cette présetation est issue.

  Les interactions du PDF sont les suivantes:
    \begin{itemize}
        \item La \hyperref[sec:TOC]{\textcolor{blue}{Table des Matières}} contient des liens cliquables vers chaque section. \pause
        \item La section \hyperref[sec:VDS]{\textcolor{blue}{Visualisation des solutions}} contient des liens vers le code générant chaque image. \pause
        \item La section \hyperref[sec:GITHUB]{\textcolor{blue}{GitHub}} contient un lien vers le dépôt associé. \pause
        \item Chaque partie interactive du PDF est ainsi facilement accessible pour le lecteur.
    \end{itemize}
\end{frame}

\begin{frame}{Présentation du projet}
    \begin{itemize}
            \item Ce document Beamer a été créé pour faciliter la présentation du projet en temps réel via Google Meet ou datashow. \pause
        \item Permet de montrer les codes, les simulations et les résultats de manière claire et interactive.
    \end{itemize}
\end{frame}



\section{Conclusion}
\begin{frame}{Conclusion générale}
    \begin{itemize}
        \item Le projet illustre le rôle essentiel des TIC dans la modélisation, la résolution et la diffusion scientifique, en particulier pour les EDP. \pause
        \item L'étude de l’équation de la chaleur illustre la traduction d’un problème physique en modèle mathématique puis résolution numérique via la méthode des différences finies. \pause
        \item MATLAB a facilité l’implémentation des schémas et la visualisation de l’évolution de la température de notre équation \pause
        \item La comparaison des solutions exacte vs solution approchée permet l'évaluation de la qualité de l’approximation et identification des limites du schéma choisi.
    \end{itemize}
\end{frame}

\begin{frame}{Aspects collaboratifs et documentaires}
    \begin{itemize}
        \item Programmation des algorithme d’approximation par MATLAB.\pause
        \item Structuration d’un espace de travail collaboratif et suivi rigoureux des modifications par Overleaf. \pause
        \item Intégration de GitHub pour centraliser les codes et documents. \pause
        \item Les TIC sont des supports essentiels pour organiser, tracer et communiquer une démarche scientifique rigoureuse.
    \end{itemize}
\end{frame}

\begin{frame}{Intérêt des TIC}
    \begin{itemize}
        \item Les TIC relient théorie et expérimentation numérique. \pause
        \item Elles améliorent la compréhension conceptuelle grâce aux visualisations interactives. \pause
        \item Elles renforcent les compétences collaboratives indispensables dans le contexte scientifique moderne.
    \end{itemize}
\end{frame}





\begin{frame}{Références}
  \bibliographystyle{plain}
  \bibliography{Références}
\cite{CanvaFeatures}
\cite{FDM_Chaleur}
\cite{GeoGebraHome}
\cite{GoogleColab}
\cite{JupyterAbout}
\cite{MathWorksNumericalMethods}
\cite{MathWorksSolutions}
\cite{MemoireChaleur}
\cite{OverleafFeatures}
\cite{PythonAbout}
\cite{TIC_LocalGlobal}
\cite{WhyGitHub}
\cite{ZoteroHome}
\end{frame}


\end{document}